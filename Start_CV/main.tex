%% start of file `template.tex'.
%% Copyright 2006-2013 Xavier Danaux (xdanaux@gmail.com).
%
% This work may be distributed and/or modified under the
% conditions of the LaTeX Project Public License version 1.3c,
% available at http://www.latex-project.org/lppl/.


\documentclass[11pt,a4paper,sans]{moderncv}        % possible options include font size ('10pt', '11pt' and '12pt'), paper size ('a4paper', 'letterpaper', 'a5paper', 'legalpaper', 'executivepaper' and 'landscape') and font family ('sans' and 'roman')

% moderncv themes
\moderncvstyle{banking}                            % style options are 'casual' (default), 'classic', 'oldstyle' and 'banking'
\moderncvcolor{red}                                % color options 'blue' (default), 'orange', 'green', 'red', 'purple', 'grey' and 'black'
%\renewcommand{\familydefault}{\sfdefault}         % to set the default font; use '\sfdefault' for the default sans serif font, '\rmdefault' for the default roman one, or any tex font name
%\nopagenumbers{}                                  % uncomment to suppress automatic page numbering for CVs longer than one page

% character encoding
\usepackage[utf8]{inputenc}                       % if you are not using xelatex ou lualatex, replace by the encoding you are using
%\usepackage{CJKutf8}                              % if you need to use CJK to typeset your resume in Chinese, Japanese or Korean

% adjust the page margins
\usepackage[scale=0.75]{geometry}
%\setlength{\hintscolumnwidth}{3cm}                % if you want to change the width of the column with the dates
%\setlength{\makecvtitlenamewidth}{10cm}           % for the 'classic' style, if you want to force the width allocated to your name and avoid line breaks. be careful though, the length is normally calculated to avoid any overlap with your personal info; use this at your own typographical risks...

% personal data
\name{Project:}{Heimdall}
\title{Mid-scale Positioning System}                               % optional, remove / comment the line if not wanted
\address{Individual Design project}{UC Davis}{California, 95616}% optional, remove / comment the line if not wanted; the "postcode city" and and "country" arguments can be omitted or provided empty
% \phone[mobile]{+1~(530)~979~4219}                   % optional, remove / comment the line if not wanted
% \phone[fixed]{+2~(345)~678~901}                    % optional, remove / comment the line if not wanted
% \phone[fax]{+3~(456)~789~012}                      % optional, remove / comment the line if not wanted
% \email{hjkang@ucdavis.edu}                               % optional, remove / comment the line if not wanted
% \email{jgtao@ucdavis.edu}
% \email{luyi326@gmail.com}
%\homepage{khjtony.github.io}                         % optional, remove / comment the line if not wanted
% \extrainfo{Mid-scale Positioning System}                 % optional, remove / comment the line if not wanted
% \photo[64pt][0.4pt]{picture}                       % optional, remove / comment the line if not wanted; '64pt' is the height the picture must be resized to, 0.4pt is the thickness of the frame around it (put it to 0pt for no frame) and 'picture' is the name of the picture file
% \quote{Some quote}                                 % optional, remove / comment the line if not wanted

% to show numerical labels in the bibliography (default is to show no labels); only useful if you make citations in your resume
%\makeatletter
%\renewcommand*{\bibliographyitemlabel}{\@biblabel{\arabic{enumiv}}}
%\makeatother
%\renewcommand*{\bibliographyitemlabel}{[\arabic{enumiv}]}% CONSIDER REPLACING THE ABOVE BY THIS

% bibliography with mutiple entries
%\usepackage{multibib}
%\newcites{book,misc}{{Books},{Others}}
%----------------------------------------------------------------------------------
%            content
%----------------------------------------------------------------------------------
\begin{document}
%-----       letter       ---------------------------------------------------------
% recipient data
\recipient{ESSC Microgrants}{UC Davis\\One Shield Ave.\\Davis, CA 95618}
\date{May 13, 2015}
\opening{Dear Officer,}
\closing{Yours faithfully,}
% \enclosure[Attached]{curriculum vit\ae{}}          % use an optional argument to use a string other than "Enclosure", or redefine \enclname
\makelettertitle

We are writing to apply for UC Davis Engineering Student Starup Microgrants Funding Program, looking for funding opportunity to make our idea to real.

\section{Team: pan-IAC}
Jonathan G. Tao: Junior year EE \& CE double majors, will graduate at June, 2016 \\
Yi Lu: Senior year EE \& CE double majors, will graduate at June, 2015 \\
Hengjiu Kang: Junior year Electrical Engineering major, will graduate at June, 2016

\section{Summer plan during 2015}
Jonathan G. Tao: TBD\\
Yi Lu: Because of being graduating during summer 2015, Yi is looking for career opportunity at California.\\
Hengjiu Kang: Registered summer session I at UC Davis. Will be in Davis till Aug.11 2015

\section{Abstract}
We believe that virtual-reality is the next generation of consumer electronics, or, the next generation of our perspectives to this world. Google glass, Hololens and Oculus Rift, all of those headset displays expand the three dimension real world a new layer: Cyber World, or Virtual World. For this future, a precise positioning system is critical to setup a universal coordinate system. We already have GPS for global positioning and iBeacon tech for in-door positioning, however we need a street-scale posistioning system serving for public users. \\
Our idea was coming from laser galvanometer and laser communication system. By modulating information into laser beam, end-device can decode its absolute coordinate. \\
The most important potential cusomers are those who are using virtual-reality devices, and this technolgy is also useful in industry, tracking products and components. \\
We have not start the project yet, but we have thought about it for years, and we decided to declare this project as our individual senior project. Additionally, Hengjiu Kang has developed a basic laser communicating prototype boards during Summer 2014 in professor Xiaoguang, Liu's lab.
The whole system has three divisions: Hardware design, software design and real product (Smart phone App). We plan to do them one by one in each quarter. 
Microgrant funds will be used to buy developing/Evaluation boards, and research components, including PCB fabrication cost. 


\makeletterclosing

\end{document}


%% end of file `template.tex'.
